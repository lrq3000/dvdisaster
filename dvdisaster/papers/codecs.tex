\documentclass[12pt,a4paper,twoside]{article}
%\usepackage{german}   	% Deutsches Wörterbuch usw.
\usepackage{times}	% Skalierbarer und lesbarer Zeichensatz
\usepackage{ucs}	% Benötigt für Eingabe von unicode-Zeichensätzen
\usepackage[utf8x]{inputenc} % Aktiviert Eingabe von unicode-Zeichensätzen
\usepackage{epsfig}	% Makros zum Einfügen von Grafiken
\usepackage{anysize}	% Makros zum Einstellen der Seitenränder
%\usepackage{makeidx}	% Makros zum Erstellen des Indexes
\usepackage{url}
\usepackage{hyperref}
\usepackage{fancyhdr}
\usepackage{listings}

\marginsize{30mm}{20mm}{20mm}{20mm}  % Seitenränder links, rechts, oben, unten
\parindent0em		% Keine amerikanische Einrückung am Anfang von Paragraphen


\pagestyle{fancy}	% Seitenstil
%\makeindex		% wird für Erstellung von Stichwortverzeichnissen benötigt

% Ende der Voreinstellungen
\newlength{\warnheight}
\newcommand{\warning}[2]{
\settoheight{\warnheight}{#2}
\addtolength{\warnheight}{#1}
\addtolength{\warnheight}{#1}
\begin{center}
 \fbox{\fbox{\hspace{#1}\rule[-#1]{0mm}{\warnheight}#2\hspace{#1}}}
\end{center}}


\newcommand{\paperversion}{{\em first draft}}

\fancyhead{}
\fancyhead[LE,RO]{page \thepage\ of \pageref{LastPage}}
\fancyhead[LO,RE]{\nouppercase{\rightmark}}
\renewcommand{\sectionmark}[1]{\markboth{#1}{#1}}
\renewcommand{\subsectionmark}[1]{}

\renewcommand{\footrulewidth}{0.4pt}
\fancyfoot{}
\fancyfoot[LE,RO]{dvdisaster codec specification}
\fancyfoot[LO]{created: \today}
\fancyfoot[RE]{\paperversion}

\begin{document}

\title{The dvdisaster Reed-Solomon Codec specification}
\author{Carsten Gnörlich\\carsten@dvdisaster.org}
\date{\today}
\maketitle
\thispagestyle{empty}
\begin{center}
\paperversion
\end{center}
%\warning{5mm}{\LARGE State: Incomplete/Evolving}

\bigskip

\begin{abstract}
This paper describes the data formats of the dvdisaster Reed-Solomon codecs
which are currently called RS01, RS02 and RS03.
The codecs create Reed-Solomon parity data to protect data stored on optical media.
Depending on the codec, parity data can either be stored in a separate file 
or be integrated with the .iso image on the same medium. 
See \url{http://dvdisaster.org}  for an overview of the dvdisaster project. 
\end{abstract}

\bigskip

{\bf Target audience.} This paper is primarily intended as a working base for the
dvdisaster developers and, when the final version has been crafted, as an implementation
guide for third party developers who wish to create and process dvdisaster error correction data.
It is {\bf neither intended nor suitable} as end-user documentation; for usage information
please refer to the online documentation at \url{http://dvdisaster.org}.

\bigskip

{\bf Prerequisites.} This paper assumes profound knowledge of coding theory and the 
underlying math. The reader is assumed to have a thorough understanding of Reed-Solomon
codes, both in theory and from an implementation viewpoint. A basic understanding
of programming in C is also assumed.

\vfill
\begin{center}
{\em 
Copyright 2008-2012 Carsten Gnörlich.
Verbatim copying and distribution of this entire article is permitted in any medium, 
provided this notice is preserved.}
\end{center}

\newpage

% Changelog

\section{Changelog}

\begin{tabular}{lp{14cm}}
V1.00 & Clarified: RS03 header does not contain copy of first CRC sector (appendix \ref{eh}). \\
      & Added $sectorsPerLayer$ field in Ecc header and CRC block format.\\
      & Added ecc file specification.\\
\end{tabular}

\newpage


% Table of Contents

\tableofcontents
\newpage

% Reed-Solomon encoding details

\documentclass[12pt,a4paper,twoside]{article}
%\usepackage{german}   	% Deutsches Wörterbuch usw.
\usepackage{times}	% Skalierbarer und lesbarer Zeichensatz
\usepackage{ucs}	% Benötigt für Eingabe von unicode-Zeichensätzen
\usepackage[utf8x]{inputenc} % Aktiviert Eingabe von unicode-Zeichensätzen
\usepackage{epsfig}	% Makros zum Einfügen von Grafiken
\usepackage{anysize}	% Makros zum Einstellen der Seitenränder
%\usepackage{makeidx}	% Makros zum Erstellen des Indexes
\usepackage{url}

\marginsize{30mm}{20mm}{20mm}{20mm}  % Seitenränder links, rechts, oben, unten
\parindent0em		% Keine amerikanische Einrückung am Anfang von Paragraphen


\pagestyle{headings}	% leerer Seitenstil (keine Seitennummern usw.)
%\makeindex		% wird für Erstellung von Stichwortverzeichnissen benötigt

% Ende der Voreinstellungen
\newlength{\warnheight}
\newcommand{\warning}[2]{
\settoheight{\warnheight}{#2}
\addtolength{\warnheight}{#1}
\addtolength{\warnheight}{#1}
\begin{center}
 \fbox{\fbox{\hspace{#1}\rule[-#1]{0mm}{\warnheight}#2\hspace{#1}}}
\end{center}}


\begin{document}

\title{The dvdisaster RS03 Reed-Solomon Codec specification}
\author{Carsten Gnörlich\\carsten@dvdisaster.org}
\date{Version 1.00}
\maketitle
\begin{center}
{\em first draft}
\end{center}
%\warning{5mm}{\LARGE State: Incomplete/Evolving}

\bigskip

\begin{abstract}
This paper describes the data format of the dvdisaster RS03 Reed-Solomon codec.
The codec creates Reed-Solomon parity data to protect data stored on optical media.
Parity data can either be stored in a separate file or be integrated with the .iso
image on the same medium. While these features were already present in the preceeding
codecs RS01 and RS02, the new RS03 codec will be fully multi-threadable.
It can therefore take full advantage of multicore processor systems and is expected
to become the default codec soon after its introduction in dvdisaster 0.80
(see \url{http://dvdisaster.org} for an overview of the dvdisaster project). 
\end{abstract}

\bigskip

{\bf Target audience.} This paper is primarily intended as a working base for the
dvdisaster developers and, when the final version has been crafted, as an implementation
guide for third party developers who wish to create and process dvdisaster RS03 data.
It is {\bf neither intended nor suitable} as end-user documentation; for usage information
please refer to the online documentation at \url{http://dvdisaster.org}.

\bigskip

{\bf Prerequisites.} This paper assumes profound knowledge of coding theory and the 
underlying math. The reader is assumed to have a thorough understanding of Reed-Solomon
codes, both in theory and from an implementation viewpoint. A basic understanding
of programming in C is also assumed.

\vfill
\begin{center}
{\em 
Copyright 2008-2010 Carsten Gnörlich.
Verbatim copying and distribution of this entire article is permitted in any medium, 
provided this notice is preserved.}
\end{center}

\newpage

% Changelog

\section{Changelog}

\begin{tabular}{lp{14cm}}
V1.00 & Clarified: RS03 header does not contain copy of first CRC sector (appendix \ref{eh}). \\
      & Added $sectorsPerLayer$ field in Ecc header and CRC block format.\\
      & Added ecc file specification.\\
\end{tabular}

\newpage


% History


% Reed-Solomon encoding details

\input{rs03-layout}

% Header formats

\appendix

\input{ecc-header}

\input{crc-block}

\input{rs-params}

\end{document}


\input{rs02}

\input{rs01}

% Header formats

\appendix

\input{ecc-header}

\input{crc-block}

\input{rs-params}
\end{document}
